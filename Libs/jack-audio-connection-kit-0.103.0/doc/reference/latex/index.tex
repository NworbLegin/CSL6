\hypertarget{index_intro}{}\section{Introduction}\label{index_intro}
JACK is a low-latency audio server, written for POSIX conformant operating systems such as GNU/Linux and Apple's OS X. It can connect several client applications to an audio device, and allow them to share audio with each other. Clients can run as separate processes like normal applications, or within the JACK server as \char`\"{}plugins\char`\"{}.

JACK was designed from the ground up for professional audio work, and its design focuses on two key areas: synchronous execution of all clients, and low latency operation.

\begin{Desc}
\item[See also:]$<$\href{http://jackit.sourceforge.net}{\tt http://jackit.sourceforge.net}$>$\end{Desc}
\hypertarget{index_jack_overview}{}\section{JACK Overview}\label{index_jack_overview}
Traditionally it has been hard if not impossible to write audio applications that can share data with each other. In addition, configuring and managing audio interface hardware has often been one of the most complex aspect of writing audio software.

JACK changes all this by providing an API that does several things:

1. provides a high level abstraction for programmers that removes the audio interface hardware from the picture and allows them to concentrate on the core functionality of their software.

2. allows applications to send and receive audio data to/from each other as well as the audio interface. There is difference in how an application sends or receives data regardless of whether it comes from another application or an audio interface.

For programmers with experience of several other audio APIs such as Port\-Audio, Apple's Core\-Audio, Steinberg's VST and ASIO as well as many others, JACK presents a familiar model: your program provides a \char`\"{}callback\char`\"{} function that will be executed at the right time. Your callback can send and receive data as well as do other signal processing tasks. You are not responsible for managing audio interfaces or threading, and there is no \char`\"{}format negotiation\char`\"{}: all audio data within JACK is represented as 32 bit floating point values.

For those with experiences rooted in the Unix world, JACK presents a somewhat unfamiliar API. Most Unix APIs are based on the read/write model spawned by the \char`\"{}everything is a file\char`\"{} abstraction that Unix is rightly famous for. The problem with this design is that it fails to take the realtime nature of audio interfaces into account, or more precisely, it fails to force application developers to pay sufficient attention to this aspect of their task. In addition, it becomes rather difficult to facilitate inter-application audio routing when different programs are not all running synchronously.

Using JACK within your program is very simple, and typically consists of just:

\begin{itemize}
\item calling \hyperlink{jack_8h_60174a8321e2d9c4e7e444e779e78e6b}{jack\_\-client\_\-open()} to connect to the JACK server.\item registering \char`\"{}ports\char`\"{} to enable data to be moved to and from your application.\item registering a \char`\"{}process callback\char`\"{} which will be called at the right time by the JACK server.\item telling JACK that your application is ready to start processing data.\end{itemize}


There is a lot more that you can do with JACK's interfaces, but for many applications, this is all that is needed. The \hyperlink{simple__client_8c}{simple\_\-client.c} example demonstrates a complete (simple!) JACK application that just copies the signal arriving at its input port to its output port. Similarly, \hyperlink{inprocess_8c}{inprocess.c} shows how to write an internal client \char`\"{}plugin\char`\"{} that runs within the JACK server process.\hypertarget{index_reference}{}\section{Reference}\label{index_reference}
The JACK programming interfaces are described in several header files:

\begin{itemize}
\item \hyperlink{jack_8h}{$<$jack/jack.h$>$} is the main JACK interface.\item \hyperlink{statistics_8h}{$<$jack/statistics.h$>$} provides interfaces for monitoring the performance of a running JACK server.\item \hyperlink{intclient_8h}{$<$jack/intclient.h$>$} allows loading and unloading JACK internal clients.\item \hyperlink{ringbuffer_8h}{$<$jack/ringbuffer.h$>$} defines a simple API for using lock-free ringbuffers. These are a good way to pass data between threads, when streaming realtime data to slower media, like audio file playback or recording.\item \hyperlink{transport_8h}{$<$jack/transport.h$>$} defines a simple transport control mechanism for starting, stopping and repositioning clients. This is described in the \hyperlink{transport-design}{JACK Transport Design} document.\item \hyperlink{types_8h}{$<$jack/types.h$>$} defines the main JACK data types.\item \hyperlink{thread_8h}{$<$jack/thread.h$>$} functions standardize thread creation for JACK and its clients.\item \hyperlink{midiport_8h}{$<$jack/midiport.h$>$} functions to handle reading and writing of MIDI data to a port\end{itemize}


In addition, the example-clients directory provides numerous examples of simple JACK clients that nevertheless use the API to do something useful. It includes

\begin{itemize}
\item a metronome.\item a recording client that can capture any number of channels from any JACK sources and store them as an audio file.\item command line clients to control the transport mechanism, change the buffer size and more.\item commands to load and unload JACK internal clients.\item tools for checking the status of a running JACK system.\end{itemize}
\hypertarget{index_porting}{}\section{Porting}\label{index_porting}
JACK is designed to be portable to any system supporting the relevant POSIX and ANSI C standards. It currently runs under GNU/Linux, Mac OS X and Berkeley Unix on several different processor architectures. If you want to port JACK to another platform, please read the \hyperlink{porting-guide}{Porting JACK} document.\hypertarget{index_license}{}\section{License}\label{index_license}
Copyright (C) 2001-2003 by Paul Davis and others.

JACK is free software; you can redistribute it and/or modify it under the terms of the GNU GPL and LGPL licenses as published by the Free Software Foundation, $<$\href{http://www.gnu.org}{\tt http://www.gnu.org}$>$. The JACK server uses the GPL, as noted in the source file headers. However, the JACK library is licensed under the LGPL, allowing proprietary programs to link with it and use JACK services. You should have received a copy of these Licenses along with the program; if not, write to the Free Software Foundation, Inc., 59 Temple Place - Suite 330, Boston, MA 02111-1307, USA.

This program is distributed in the hope that it will be useful, but WITHOUT ANY WARRANTY; without even the implied warranty of MERCHANTABILITY or FITNESS FOR A PARTICULAR PURPOSE. See the GNU General Public License for more details. 